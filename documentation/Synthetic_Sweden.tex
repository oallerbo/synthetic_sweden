\title{Synthetic Sweden}
\author{Oskar Allerbo}
\documentclass[12pt]{article}
\begin{document}
\section{Baseline Synthetic Population}
\begin{itemize}
\item
IPF is done on household level for attributes age, gender, marital status and
income, to generate a joint distribution.
\item
IPF takes as input an $n$-dimensional joint distribution and $n$ one dimensional
marginal distributions and then modifies the joint distribution to match the
marginal distributions. The new joint distribution is the maximum liklihood
estimation given that the margins shall fit. 
\item
Households are picked from the PUMS according to the distribution above. The
individuals come with the household.
\end{itemize}

\section{Locations}
\begin{itemize}
\item
It is known how many households there shall be within the geographical unit.
\item
Households are placed along roads, with single person households more likely
to be placed along smaller streets.
\item
Households are assigned to households locations.
\item
Activity locations are assigned somehow.
\item
People are assigned locations for work/school where the probability to pick
work/school location $j$, given home location $i$ is $P(j|i) =
\frac{c_je^{b_mT_{ij}}}{\sum_{j'} c_{j'}e^{b_mT_{ij'}}}$, where $c_j$ is the capacity
of location $j$, $T_{ij}$ is the travel time between $i$ and $j$ and $b_m$ is a
travel mode coefficient.
\item
To speed up calculations only locations within a certain radius of the home
location are considered.
\item
People are assigned locations for other activities, taking place between home
and work/shool with probability to pick location $k$ given by $P(k|j,i) =
\frac{c_ke^{b_{m1}T_{jk} + b_{m2}T_{ik}}}{\sum_{k'} c_ke^{b_{m1}T_{jk'} +
b_{m2}T_{ik'}}}$, with coefficients as before.
\end{itemize}

\section{Schedules}
\begin{itemize}
\item
There is survey data with real people with real schedules, doing activities like being
at home, being at work, being in a shop and so on.
\item
Synthetic people are compared to the survey people and assigned the schedule of
the survey person that they are most similar.
\item
Only the activities are assigned, the activity locations used are the ones
perviously assigned to the synthetic people.
\item
Similarity is measured in the following way:
\begin{itemize}
\item
For each survey person a vector with the sum of the total times spent at different
activities is created: $\vec{Y} = (T_{\textrm{home}}, T_{\textrm{work}},
T_{\textrm{shop}}, ...)$
\item 
All the attributes of the survey persons are put in a vector $\vec{X}$, which
can contain a mix of continueous and catogrical variables. Maybe the age is
continues, while the income is in one of three bins, then $\vec{X}$ might look
like $(37, 0, 1, 0, ...)$ for a 37 year old with middle high income.
\item
Using linear regression a model is calculated which models $\vec{Y_i} =
\vec{X_i}\vec{\beta_i}$, for each activity $i$.
\item
Using this model times spent at activities are calculated for all synthetic
persons and all survey persons. Note that the survey person now get new, fitted,
times. This to make sure that two person with identical attributes also will
have identical times, since the times will be used to measure the person-person
distances.
\item
The distance between two persons is defined as the Mahalanobis distance between
their $\vec{Y}$ vectors. The Mahalanobis distance is like the Euclidean
distance, but with all components normalized.
\item
The distance between two households is defined in the following way:
\begin{itemize}
\item
For each person in the first household the distance to each of the persons in
the other household is calculated and the minimum of these distances is
remembered, i.e. the distance to the closest person in the other household.
\item
As the distance between the two households the biggest distance of those from
above, i.e. the maximum of the minimum of the person-person distances.
\end{itemize}
\end{itemize}
\item
Now assigning schedules to synthetic persons is done in the following way:
\begin{itemize}
\item
A synthetic household is selected.
\item
The most similar survey household is found.
\item
Each person in the synthetic household is assigned the schedule of the person in
the survey household they are most similar. Note that this might lead synthetic
persons within the household having identical schedules.
\end{itemize}
\end{itemize}

\section{Questions}
\begin{itemize}
\item
Is IPF done on household level and then individuals are picked together with
households?
\item 
How are person things, like sex, measured on household level?
\item
What is the PUMA size in Sweden?
\item
When do you stop picking, when the amount of individuals are correct or the
amount of households?
\item
Are the individuals somehow changed or is the same individual present many
times?
\item
How is age transformed from bin to actual age? Uniform distribution?
\item
When a household is sampled from a bin, is there any process to pick different
households at different samples?
\\
\item
What are the geographical units in Sweden? län?
\item
What do the different hloctypes mean? 1, 2, 4, 49.
\item
When households are placed at household locations, are the household properties
taken into account when assigning it to a hloctype?
\item
How are activity locations assigned?
\item 
How are activity locations chosen when there are multiple activities between
anchor locations?
\end{itemize}

\section{Attributes}
\subsection{agep}
\begin{itemize}
\item All numbers in [0, 99] represented
\end{itemize}

\subsection{sex}
\begin{itemize}
\item 1, 2
\item E-mail from Eric:
\begin{itemize}
\item 1 Male
\item 2 Female
\end{itemize}
\end{itemize}

\subsection{maritalb}
\begin{itemize}
\item 1, 2, 4, 5, 6
\item maritala: LEGAL MARITAL STATUS [1]
\begin{itemize}
\item 01 Married 
\item 02 In a civil partnership 
\item 03 Separated (still legally married) 
\item 04 Separated (still in a civil partnership) 
\item 05 Divorced 
\item 06 Widowed 
\item 07 Formerly in civil partnership, now dissolved 
\item 08 Formerly in civil partnership, partner died 
\item 09 Never married and never in civil partnership 
\item 77 Refusal 
\item 88 Don't know 
\item 99 No answer 
\end{itemize}
\end{itemize}

\subsection{tporgwk}
\begin{itemize}
\item -1, 1, 2, 3, 4, 5, 6, 66, 88
\item WHAT TYPE OF ORGANISATION WORK/WORKED FOR [1]
\begin{itemize}
\item 01 Central or local government 
\item 02 Other public sector (such as education and health) 
\item 03 A state owned enterprise 
\item 04 A private firm 
\item 05 Self employed 
\item 06 Other                                                  
\item 66 Not applicable 
\item 77 Refusal                                            
\item 88 Don’t know 
\item 99 No answer
\end{itemize}
\end{itemize}

\subsection{iscoco}
\begin{itemize}
\item 233 numbers, -1, 100, 66666, 99999, and some numbers in [1110, 9320].
\item OCCUPATION, ISCO88 (COM) [1]
\begin{itemize}
\item 66666 Not applicable 
\item 77777 Refusal
\item 88888 Don’t know 
\item 99999 No answer 
\end{itemize}
\item
ISCO = International Standard Classification of Occupations. ISCO-88
Ex: 9212=Livstock farm labourers. [3]
\end{itemize}

\subsection{hlthmp}
\begin{itemize}
\item -1, 1, 2, 3
\item hlthhmp: HAMPERED IN DAILY ACTIVITIES BY ILLNESS/DISABILITY/INFIRMARY/MEN
TAL PROBLEM [1]
\begin{itemize}
\item 1 Yes a lot 
\item 2 Yes to some extent 
\item 3 No                                                          
\item 7 Refusal                                            
\item 8 Don’t know 
\item 9 No answer 
\end{itemize}
\end{itemize}

\subsection{edlvdse}
\begin{itemize}
\item -1, [1, 20], 8888, 9999
\item edlvdxx: HIGHEST LEVEL OF EDUCATION, [COUNTRY] [2]
\begin{itemize}
\item 5555 Other
\item 7777 Refusal
\item 8888 Don’t know                                
\item 9999 No              
\end{itemize}
\end{itemize}

\subsection{income}
\begin{itemize}
\item -1, [1, 10]
\end{itemize}

\subsection{hloctype}
\begin{itemize}
\item 1, 2, 9, 49
\end{itemize}

\subsection{activity types}
\begin{itemize}
\item E-mail from Eric:
\begin{itemize}
\item 1 home
\item 2 work
\item 3 shopping
\item 4 other
\item 5 school
\item 6 college
\end{itemize}
\end{itemize}

\begin{itemize}
\item
1 https://www.europeansocialsurvey.org/docs/round4/survey/ESS4\_data\_protocol\_e01\_2.pdf
\item
2 https://www.europeansocialsurvey.org/docs/round5/survey/ESS5\_data\_protocol\_e02\_4.pdf
\item
3 http://www.ilo.org/public/english/bureau/stat/isco/isco08/index.htm
\end{itemize}
\end{document}
